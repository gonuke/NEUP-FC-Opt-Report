\section*{Executive Summary}
%% 3. Provide an executive summary, which includes a discussion of 1) how the
%%    research adds to the understanding of the area investigated; 2) the
%%    technical effectiveness and economic feasibility of the methods or
%%    techniques investigated or demonstrated; or 3) how the project is otherwise
%%    of benefit to the public. The discussion should be a minimum of one
%%    paragraph and written in terms understandable by an educated layman.

Alternatives nuclear cycles offer a number of possible advantages over the
current once-through fuel cycle, but there are lingering uncertainties in how
to assess those advantages and unclear transition paths to arrive at those
alternatives.  The field of nuclear fuel cycle simulation provides tools for
investigating these transitions, studying their technical challenges and
identifying their broad societal benefits.  Most fuel cycle simulation relies
on the analyst to propose a specific deployment plan for different
technologies in the future, and then use the simulation tools to measure the
consequences.  Any attempt to optimize has generally relied on manual
iteration to revise the deployment plan and reassess the transition.

This work seeks to introduce automated optimization into fuel cycle
simulation, specifically using the \Cyclus{} fuel cycle simulation platform.
At the system-level, this consists of implementing an optimization driver that
can automatically assess the performance of a single deployment plan and adapt
it to seek a better alternative.  In addition to identifying a preferred
optimization algorithm, some consideration must also be given to the structure
of the decision space and the formulation of the objective function.  While
there is too much uncertainty in many nuclear fuel cycle performance metrics
for a complete single objective function to be defined, it is possible to
define some useful functions.

In order to support the flexibility of \Cyclus{} in responding to the needs of
system-level optimization, a form of optimization must also be introduced at
the market-level.  Individual facilities trade in nuclear materials using a
market paradigm in which requests for material are matched with offers of
material.  The consumers of material have the freedom to determine the best
offers based on algorithms that meet their individual needs, whether physics,
economics or otherwise.  Such a market mechanism allows each facility to
respond freely, and possibly with nuance and richness, to perturbations
introduced to the simulation by the system-level optimization process.  This
mechanism also preserves the ability for individual facility models to operate
without any special knowledge of the internal state or decisions of other
facilities.  Their only interaction is in the form of requests and offers.

Although the consumers retain the agency to make the final preference
decisions, it is important for the supplier to have some input to those
decisions.  It is also valuable for suppliers to have a mechanism to assess
how any given offer will be received by the potential consumers.  Tailored
offers will help maximize the performance of the market as a whole.  Both of
these capabilities have been introduced as part of the otherwise
consumer-centric market model.

Finally, the system-level optimization has been extended to perform nested
optimization under postulated disruptions to the deployment plan, thus
identifying deployment strategies that hedge against the impact of those
disruptions.


