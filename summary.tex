\section{Summary}

This project identified two primary goals related to the introduction of
optimization capability into the \Cyclus{} fuel cycle simulation ecosystem:
\begin{enumerate}
\item optimization at the market-level for individual commodity trades at each time step, and 
\item optimization at the system-level to identify the best deployment
  histories over a fuel cycle transition.
\end{enumerate}
\noindent Both of these goals were successfully accomplished with minor variations from
originally proposed research plan.

\subsection{Market-level Optimization}

The mechanism for matching resource requests made by some facilities with
resource offers made by other facilities was finalized and implemented as a
variation of a classic network transportation problem.  Specific additions
were made to account for the fungibility of some nuclear materials, the desire
for exclusive supply arrangements, and the possibility for constraints that
are functions of both resource quality and quantity.  This approach also
preserves the important feature of retaining full agency within each facility
for determining its preference among the offers that can satisfy its requests.

The specific task of assessing different optimization algorithms was completed
with a comprehensive comparison of a fast heuristic market matching algorithm
with more rigorous linear programming algorithms provided by \gls{COIN-OR}.
The heuristic matching algorithm was found to be sufficient in many cases,
although it is not yet clear how to determine that it will be sufficient for
any particular problem.  Although the total objective function value of
different material flow solutions may vary, such variations do not always
represent substantially different flows.

The specific task of introducing objective function callbacks to allow
suppliers to determine what preference each consumer will assess for its
offers, and therefore tailor the offers to maximize the likelihood of
matching.  This feature both adds efficiency to the market mechanisms by
improving the quality of individual offers and/or reducing the total number of
offers necessary to reach an optimal solution.

The single deviation from the original research plan was that realistic
economic value functions were not implemented in place of preference.  While
it is possible for each facility archetype to rely on realistic economic
values to assess its preference for individual trades in the current
implementation of \Cyclus{}, insufficient certainty in economic models exists
to justify incorporating this in specific facility archetypes being delivered
formally as part of the standard software.

\subsection{System-level Optimization}

A system for seeking and identifying optimal deployment histories was
developed based on black-box optimization techniques.  A PSwarm algorithm was
found to be the most effective at converging on optimal deployment histories,
following comparison with a variety of open source algorithms that satisfied
the requirements for optimization of this kind of problem.  The formulation of
the decision space was found to be important to improve the rate of
convergence.  A master-worker paradigm was implemented to take advantage of
high throughout and/or cloud computing resources necessary to accomplish the
$O(10^5)$ \Cyclus{} simulations necessary in this kind of optimization.

The optimization system was demonstrated using the EG23 fuel cycle transition
defined by the Fuel Cycle Options campaign.  While the FCO campaign relied on
manual iteration to identify an optimum deployment history, the optimization
system was able to automatically arrive at an equivalent deployment history.
Allowing for some flexibility in the overall power constraint allowed the
system to find a more optimal solution that completes the transition more
quickly.

One contribution beyond the original research plan was to use this
optimization system to compare the impacts of different modeling choices,
namely the difference between modeling individual facilities and fleets of
facilities, and the difference between 1 month and 3 month time
steps\citeprod{rwc_fleet}.

One substantial contribution beyond the original research plan was the
adaptation of the optimization system for identifying hedging strategies in
the face of possible disruptions to the planned fuel cycle transition.  A
conceptual methodology was designed that relies on nested optimization, and
implemented with an approximation to one level of optimization to reduce the
overall computational burden to a reasonable level.






%% 4. Provide a comparison of the actual accomplishments with the goals and
%%    objectives of the project.


%% 6. Identify products developed under the award and technology transfer activities, such as:
%%    a. Publications (list journal name, volume, issue), conference papers, or
%%       other public releases of results. If not provided previously, attach or
%%       send copies of any public releases to the DOE Program Manager identified
%%       in Block 15 of the Assistance Agreement Cover Page;
%%    b. Web site or other Internet sites that reflect the results of this
%%       project;
%%    c. Networks or collaborations fostered;
%%    d. Technologies/Techniques;
%%    e. Inventions/Patent Applications, licensing agreements; and
%%    f. Other products, such as data or databases, physical collections, audio
%%       or video, software or netware, models, educational aid or curricula,
%%       instruments or equipment.
