\section{Introduction}

The goal of this project was to introduce optimization capabilities into the
\Cyclus{} fuel cycle simulator at two different levels of the simulation
capability.  

\subsection{\Cyclus{} Fundamentals}

\Cyclus{} is a nuclear fuel cycle simulation platform with a primary design
goal of flexibility to accommodate future innovations in nuclear fuel cycles.
This flexibility is achieved, primarily, through an agent-based paradigm in
which different models for agent behavior can be introduced through runtime
plugin modules.  The most important outcome of this paradigm is that new
behaviors can be introduced without having to change the fundamental platform,
thus allowing for better understanding of the impacts of those new behaviors.
Such new behaviors can represent different levels of fidelity for modeling
similar facilities, or entirely novel facilities that require different
modeling options.

A \Cyclus{} simulation consists of a deployment plan for a series of
facilities, possibly over many decades, with each facility represented by a
particular behavior model.  Each facility has a set of commodities that it
consumes and/or supplies, and a market for each commodity exists to facilitate
the trade of those commodities.  Commodities are traded in discrete quanta,
each of which has a specific quantitiy (often measured in mass) and a specific
quality (e.g. isotopic composition).  At each time step, consumers issue
requests for resources while suppliers issue offers of resources. A market
model uses some mechanism to determine which requests will be matched with
which offers, and resources are exchanged among facilites.  A more complete
description of the \Cyclus{} platform, including some of the work described
here, can be found in \cite{ref:cyclus_fundamentals}.

\subsection{Optimization Needs}

The first target of optimization is at the level of individual commodity
markets during each time step.  Fundamental to the design of \Cyclus is the
ability for different facilities (often referred to as agents) to trade
discrete quanta of resources in the nuclear fuel cycle.  Initial
implementations of \Cyclus relied on an omniscient market agent that would
collect all the offers and requests and use some algorithm to determine which
trades would take place.  While this did allow for market like behavior, it
did not allow for different agents to value the same discrete quantum of
material differently.  Market-level optimization was implemented in the form
of the \gls{DRE} with the primary purpose of supporting agent-based valuation
of indvidual resource bids, accounting for flexibility in the behavior of the
facility and fungibility among isotopes within a particular commodity.  As
will be seen, this encapsulation of the resource bid valuation entirely within
each agent promotes the fundamental flexibility of \Cyclus.

By itself, the consumer-centric \gls{DRE} left one gap in the efficient
implementation of commodity markets: suppliers had no knowledge of how the
consumers would value their bids and therefore no ability to customize their
bids to the consumer.  This was rememdied by extending the contents of a
request for resources to also include a mechanism for suppliers to query the
consumers for the value of a particular offer, so-called objective function
callbacks.  One original aim that proved unrealistic was the introduction of
realistic economic value functions in the context of these callback functions.
The ability to define the value of an resource offer in purely economic terms
is an appealing notion.  However, the remains too much uncertainty in the
various components of the costs of most fuel cycle facilities and processes to
provide credible estimates of the economic value of any single resource offer.
As will be discussed, the implementation of the objective function callbacks
does not preclude the introduction of purely economic valuation, but none were
explicitly provided as part of this work.

The second target of optimization is at the level of an entire simulation.  A
single \Cyclus simulation can simulate the deployment, interaction and
decommissioning of a set of facilities over time, and provide metrics for the
performance of those facilities as well as for the overall fuel cycle.  It is
frequently interesting, however, to find a particular set of deployment,
interaction and decommissioning characteristics that will optimize some
combination of those metrics.  Given the emergent behavior that can arise from
an agent-based simulation, stochastic optimization techniques are the best
choice.  A swarm-based algorithm was implemented and demonstrated on
representative nuclear fuel cycle transition scenarios.  This same algorithm
was then also used to explore the limitations of different modeling paradigms
in achieving the same optimum solution.  Finally, a nested optimization was
carried out to identify best hedging scenarios given some the likelihood of
disruption in the fuel cycle during transition.

While economic valuation of individual resource offers at the market level
there is also interest in determining an economic/financial metric for the
performance of an entire fuel cycle, particularly during transition.  Such a
metric would permit system-level optimization that focused entirely/primarily
on the economic performance over time.  Some work was done to consider the
economic metrics for complete nuclear fuel cycles, particularly during
transition.

\subsection{About this Report}

This report documents the implementation, testing, characterization and
demonstration of each of these features.  Section \ref{section:market}
describes the \gls{DRE} and the addition of objective function callbacks.
Section \ref{section:system} describes the fundamental optimization capability
and two advanced demonstrations of its use in fuel cycle simulations.  A brief
summary of the work on economic objective functions is also provided.

