\section{Market-level Optimization}\label{section:market}

The original design of \Cyclus required some mechanism to match the consumer
requests with supplier offers, based on both their quantity and quality
(e.g. isotopic composition).


This section describes the capability that allows \Cyclus to match consumer
requests with supplier offers while consistent with the fundamental goals of
the \Cyclus framework.  In particular, this capability must support the
following.

\vspace{1em}
\noindent\textbf{Flexibility}: In order to support a maximum level of
flexibility in \Cyclus, each agent must operate as a \textit{black box}, with
no assumed knowledge of the capabilities or current state of other agents.
This permits individual facilities to be represented by plugins that can be
swapped individually without imposing restrictions on other agents in the
system.  The market clearing mechanism must be able to accept any request for
resources, regardless of its ability to be satisfied by the system.

\vspace{1em}
\noindent\textbf{Fungibility}: Since different fissile nuclides can fill
similar roles in the fuel cycle, resource offers may differ in composition but
still satisfy the needs of the consumer.  The market clearing mechanism must
be able to accept any offer of resources, without judgement on whether or not
it meets the goals of the corresponding request.

\vspace{1em}
\noindent\textbf{Agency}: Given the prior two characteristics, the final
ability to determine the details of a specific request, a specific offer, or
which offer is the most suitable match to a specific request must rest solely
in the hands of the facilities making those requests and offers.  The market
clearing mechanism may not impose any problem-wide value judgement on the
transactions.
\vspace{1em}

A market clearing mechanism that satisfies these high level requirements,
known as the \gls{DRE}, was designed and implemented.  \ref{subsection:dre}
first discusses the conceptual design of the \gls{DRE}, followed by a formal
definition of this system as a mixed-integer linear program to solve a
modified network flow problem.  Different solvers were studied to assess the
trade-off between accuracy and efficiency.  A number of sample problems are
presented to demonstrate the capability of the \gls{DRE}.

Although the \gls{DRE} provides the necessary components for market clearing
in \Cyclus, it does not lead to the most efficient results.  

In this \gls{DRE} design, consumers issue requests with no \textit{a
  priori} certainty that there will be a supplier that will be physically able
to satisfy that request.  A user will want to ensure that other agents exist
in the simulation that are able to supply their consumers, but the developer
of the agent plugin does not have such restrictions.





While a simulation in which no consumer gets bids from
a supplier is not interesting, occasional failure of supply is an interesting
result often sought in studying fuel cycle transitions.


  \Cyclus is designed to support new agents as runtime plugins To support the plugin architecture of \Cyclus, there must
  be no requirement for \textit{a priori} knowledge of possible trading
  relationships.






In section 

In secction \ref{subsection:callback}, the introduction of objective function
callbacks is described as a mechanism to increase the overall efficiency of
the system by allowing suppliers to better match the needs of their consumers.

\subsection{Dynamic Resource Exchange}\label{subsection:dre}

\subsubsection{Conceptual Design}

At the conceptual level, the \gls{DRE} implements the following communication
cycle at each time step in a \Cyclus simulation:
\begin{enumerate}
\item All consumers broadcast their requests to all suppliers.  Each request
  describes the desired quantity and quality (isotopic composition).
\item Suppliers respond to each consumer request with an offer to supply
  material.  Each supplier has the agency to decide whether or not to respond,
  and if so, how to respond.  Each offer describes an available quantity and
  quality.
\item Consumers use their own valuation process to rank the offers using the
  notion of preference.
\item All preferences are collected to be solved by a system that maximizes
  the total preference in the system.
\item The solution identifies specific resource trades that are then carried
  out among the facilities/agents.
\end{enumerate}

This conceptual design is derived from a need to support the following \Cyclus
design goals.




\subsubsection{Mathematical Formulation}
\subsubsection{Solution Engines}
\subsubsection{Demonstration Problems}
\subsection{Objective Function Callbacks}\label{subsection:callback}

\subsubsection{Motivation}
\subsubsection{Implementation}
\subsubsection{Demonstration Problem}

